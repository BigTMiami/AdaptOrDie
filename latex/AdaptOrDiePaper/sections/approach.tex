Our project had three main phases: reproduce initial pre-training results, replicate the results using an adapter architecture and explore potential novel benefits in using adapters.

\subsection{Reproducing Results}

Our first goal was to reproduce results from Gururangan 2020 \cite{gururangan2020don} for Domain Adaptive Pre-Training (DAPT) and Task Adaptive Pre-Training (TAPT) for at least one domain.  This would allow us to create a nuts and bolts understanding of the pre-training process and ensure we could make it work.

\subsubsection{DAPT Training}
Our first unexpected challenge related to data.  Our proposal research had shown that this paper had an online repository that held both the code and data used for the paper.  While the Task data was available as expected, the Domain data, which was much larger, was not.  We decided to focus on the Amazon Review data because we were able to identify what we think is a similar source and it was the smallest of the Domain datasets.  Gururangan 2020 identifies  He, McAuley 2016 \cite{amazonData} as the Domain data source, which led to Ni, Li, McAuley \cite{amazonData_Part_2} paper, which in turn has a website with Amazon reviews.  

Gururangan 2020 describes 24.76 million reviews in their dataset.  McAuley's site contained several versions, none of which matched this.  We chose a filtered set ("5-core") of the 2018 dataset, as we thought it was most likely closest to the original. It contained 37 different categories with a total of 75 million reviews.  We sampled each category proportionally to get to 25 million then shuffled the data.

A key challenge we had anticipated was the computational load of running LLM models.  The roberta-base model we used contains ~124 million parameters.  While relatively small by today's standards, this is still quite large when performing training.  Gururangan 2020 used a large batch size for DAPT pre-training, which required the use of gradient accumulation.  While this allows training to fit into memory, smaller batch sizes creates longer the run times.  The pre-training performs Masked Language Modeling, which does not use labels, so we did have one option.  The context window for the roberta-base model is 512 tokens and the average review was 87 tokens long.  Each training batch was mostly empty, but used the same amount of memory for computing gradients for backprop.  We could put almost 6 reviews on average in each row, increasing training efficiency by almost a factor of 6, while still exposing the model to the same number of reviews. 

One other improvement we were able to use on the initial reproduction was the use of PyTorch compilation.  Since version 2.0, PyTorch allows for compilation of the computation graph, which in our case almost doubled performance, even with the overhead associated with the compilation.  

This brings us to our second unexpected challenge, which was the effective age of the Gururangan 2020 codebase.  The versions of tools needed to run the allennlp tool set \cite{gardner-etal-2018-allennlp} are quite old and we were not able to install them on the Google Colab platform \cite{googlecolab}.  One of our team members had access to a local environment with some computational resources where she was able to make adjustments to allow for installation, but it was tedious.  This highlighted the importance of using a robust set of platform tools that will be kept current and working. 

The rest of the group began work using the transformers platform \cite{transformers}  to reproduce the results.  The transformers platform allows for standardization of many low value tasks, such as tokenization, data collation and basic training  evaluation loops.  The Huggingface hub allows for models, training parameters, tokenizations and more to be stored in a common format on the internet with open access.  This is very valuable for group research, replication of previous works and addition of new enhancements. We also used the datasets platform  \cite{datasets} to easily manage our datasets.  We could pre-process the raw dataset into tokens and upload so that each of the team members could have access, especially on the Google Colab platform.

\subsubsection{Classification}

For classification (and TAPT), we were able to use three of the same datasets used in the original paper:  Amazon Helpfulness, IMDB Review Sentiment and Citation Intent.  We converted the text to tokens using the transformer roberta-base tokenizer and loaded them to the HuggingFace Hub.  

\textbf{Figure \ref{fig:dataset_analysis}} shows two key aspects of these datasets.  The primary dataset we used was Amazon Helpfulness which is both the largest (seen on the left part of the figure) as well as very imbalanced, seen on the right side of the figure.  We can also see that the Citation Intent dataset is quite imbalanced.  Gururangan 2020 uses an F1 Macro score for evaluation, which is a good measure for imbalanced data.  The "Macro" setting weights each class equally.

We were able to then perform classification using the transformers.  We performed the first classification using the roberta model.  For the DAPT classification, we would remove the Masked Language Modeling (MLM) head used in pre training and replace it with a classification head.  We were able to reproduce a positive DAPT pre training impact, which was encouraging, but the impact was very modest.  The extreme times for pre training and classification were limiting for trying different experiments.

\subsubsection{TAPT Training}

For TAPT training, instead of using domain data for pretraining, the data that will be used in the classification task is used for masked language model fine tuning.  Just as we did for domain pretraining, we created a set of condensed datasets from the Task data to speed pretraining.  The Task pretraining was quicker than domain pretraining due to the smaller data size, but classification still took a long time.  

We were able also able to reproduce positive TAPT results, as shown in the Experiments section.

\subsection{Replicating using Adapters}

Having been able to reproduce the basic findings for DAPT using Amazon reviews and TAPT, we felt comfortable that we had good data and effective models and training.  The next goal was to implement this using an Adapter architecture. As described in the Introduction, we chose to work with two adapter architectures at this point, sequential bottleneck and UniPelt. 

We were able to somewhat replicate Gururangan 2020 using the adapter architectures, as shown in the Experiment and Results section.  Computational cost and time was a significant issue, shaping the final phase of our work.

\subsection{Adapter Exploration}
While we were able to replicate positive pretraining effects using adapters, we were very limited in our ability to explore different configurations.  We decided to focus on reduced size dataset just for TAPT, with the focus on using only 5k reviews in the "Micro Help" dataset seen in \textbf{Figure \ref{fig:dataset_analysis} }.   We liked that Task Adaptive Pre Training would be available to all users for classification tasks, without having to consider finding the right domain dataset.  We also wanted to explore the smallest amount of pretraining that would produce a positive TAPT impact.  This joined our interest in a more tractable experimentation framework with the general users interest in improving results with the least amount of effort.  Finally, we thought we would use an "intermediate" adapter architecture, parallel bottleneck,  with the bottlenecks in parallel instead of sequentially, as proposed in He et al. 21 \cite{he2022unified}.  We used the same reduction factor as with the sequential bottleneck adapter, so it contained the same amount of parameters as seen in \textbf{Table \ref{tab:adapter_parameters}}.







